\documentclass[]{article}
\usepackage{lmodern}
\usepackage{amssymb,amsmath}
\usepackage{ifxetex,ifluatex}
\usepackage{fixltx2e} % provides \textsubscript
\ifnum 0\ifxetex 1\fi\ifluatex 1\fi=0 % if pdftex
  \usepackage[T1]{fontenc}
  \usepackage[utf8]{inputenc}
\else % if luatex or xelatex
  \ifxetex
    \usepackage{mathspec}
  \else
    \usepackage{fontspec}
  \fi
  \defaultfontfeatures{Ligatures=TeX,Scale=MatchLowercase}
\fi
% use upquote if available, for straight quotes in verbatim environments
\IfFileExists{upquote.sty}{\usepackage{upquote}}{}
% use microtype if available
\IfFileExists{microtype.sty}{%
\usepackage{microtype}
\UseMicrotypeSet[protrusion]{basicmath} % disable protrusion for tt fonts
}{}
\usepackage[margin=1in]{geometry}
\usepackage{hyperref}
\hypersetup{unicode=true,
            pdftitle={California Housing Prices},
            pdfauthor={Diego Alejandro Mernies},
            pdfborder={0 0 0},
            breaklinks=true}
\urlstyle{same}  % don't use monospace font for urls
\usepackage{color}
\usepackage{fancyvrb}
\newcommand{\VerbBar}{|}
\newcommand{\VERB}{\Verb[commandchars=\\\{\}]}
\DefineVerbatimEnvironment{Highlighting}{Verbatim}{commandchars=\\\{\}}
% Add ',fontsize=\small' for more characters per line
\usepackage{framed}
\definecolor{shadecolor}{RGB}{248,248,248}
\newenvironment{Shaded}{\begin{snugshade}}{\end{snugshade}}
\newcommand{\KeywordTok}[1]{\textcolor[rgb]{0.13,0.29,0.53}{\textbf{#1}}}
\newcommand{\DataTypeTok}[1]{\textcolor[rgb]{0.13,0.29,0.53}{#1}}
\newcommand{\DecValTok}[1]{\textcolor[rgb]{0.00,0.00,0.81}{#1}}
\newcommand{\BaseNTok}[1]{\textcolor[rgb]{0.00,0.00,0.81}{#1}}
\newcommand{\FloatTok}[1]{\textcolor[rgb]{0.00,0.00,0.81}{#1}}
\newcommand{\ConstantTok}[1]{\textcolor[rgb]{0.00,0.00,0.00}{#1}}
\newcommand{\CharTok}[1]{\textcolor[rgb]{0.31,0.60,0.02}{#1}}
\newcommand{\SpecialCharTok}[1]{\textcolor[rgb]{0.00,0.00,0.00}{#1}}
\newcommand{\StringTok}[1]{\textcolor[rgb]{0.31,0.60,0.02}{#1}}
\newcommand{\VerbatimStringTok}[1]{\textcolor[rgb]{0.31,0.60,0.02}{#1}}
\newcommand{\SpecialStringTok}[1]{\textcolor[rgb]{0.31,0.60,0.02}{#1}}
\newcommand{\ImportTok}[1]{#1}
\newcommand{\CommentTok}[1]{\textcolor[rgb]{0.56,0.35,0.01}{\textit{#1}}}
\newcommand{\DocumentationTok}[1]{\textcolor[rgb]{0.56,0.35,0.01}{\textbf{\textit{#1}}}}
\newcommand{\AnnotationTok}[1]{\textcolor[rgb]{0.56,0.35,0.01}{\textbf{\textit{#1}}}}
\newcommand{\CommentVarTok}[1]{\textcolor[rgb]{0.56,0.35,0.01}{\textbf{\textit{#1}}}}
\newcommand{\OtherTok}[1]{\textcolor[rgb]{0.56,0.35,0.01}{#1}}
\newcommand{\FunctionTok}[1]{\textcolor[rgb]{0.00,0.00,0.00}{#1}}
\newcommand{\VariableTok}[1]{\textcolor[rgb]{0.00,0.00,0.00}{#1}}
\newcommand{\ControlFlowTok}[1]{\textcolor[rgb]{0.13,0.29,0.53}{\textbf{#1}}}
\newcommand{\OperatorTok}[1]{\textcolor[rgb]{0.81,0.36,0.00}{\textbf{#1}}}
\newcommand{\BuiltInTok}[1]{#1}
\newcommand{\ExtensionTok}[1]{#1}
\newcommand{\PreprocessorTok}[1]{\textcolor[rgb]{0.56,0.35,0.01}{\textit{#1}}}
\newcommand{\AttributeTok}[1]{\textcolor[rgb]{0.77,0.63,0.00}{#1}}
\newcommand{\RegionMarkerTok}[1]{#1}
\newcommand{\InformationTok}[1]{\textcolor[rgb]{0.56,0.35,0.01}{\textbf{\textit{#1}}}}
\newcommand{\WarningTok}[1]{\textcolor[rgb]{0.56,0.35,0.01}{\textbf{\textit{#1}}}}
\newcommand{\AlertTok}[1]{\textcolor[rgb]{0.94,0.16,0.16}{#1}}
\newcommand{\ErrorTok}[1]{\textcolor[rgb]{0.64,0.00,0.00}{\textbf{#1}}}
\newcommand{\NormalTok}[1]{#1}
\usepackage{longtable,booktabs}
\usepackage{graphicx,grffile}
\makeatletter
\def\maxwidth{\ifdim\Gin@nat@width>\linewidth\linewidth\else\Gin@nat@width\fi}
\def\maxheight{\ifdim\Gin@nat@height>\textheight\textheight\else\Gin@nat@height\fi}
\makeatother
% Scale images if necessary, so that they will not overflow the page
% margins by default, and it is still possible to overwrite the defaults
% using explicit options in \includegraphics[width, height, ...]{}
\setkeys{Gin}{width=\maxwidth,height=\maxheight,keepaspectratio}
\IfFileExists{parskip.sty}{%
\usepackage{parskip}
}{% else
\setlength{\parindent}{0pt}
\setlength{\parskip}{6pt plus 2pt minus 1pt}
}
\setlength{\emergencystretch}{3em}  % prevent overfull lines
\providecommand{\tightlist}{%
  \setlength{\itemsep}{0pt}\setlength{\parskip}{0pt}}
\setcounter{secnumdepth}{0}
% Redefines (sub)paragraphs to behave more like sections
\ifx\paragraph\undefined\else
\let\oldparagraph\paragraph
\renewcommand{\paragraph}[1]{\oldparagraph{#1}\mbox{}}
\fi
\ifx\subparagraph\undefined\else
\let\oldsubparagraph\subparagraph
\renewcommand{\subparagraph}[1]{\oldsubparagraph{#1}\mbox{}}
\fi

%%% Use protect on footnotes to avoid problems with footnotes in titles
\let\rmarkdownfootnote\footnote%
\def\footnote{\protect\rmarkdownfootnote}

%%% Change title format to be more compact
\usepackage{titling}

% Create subtitle command for use in maketitle
\providecommand{\subtitle}[1]{
  \posttitle{
    \begin{center}\large#1\end{center}
    }
}

\setlength{\droptitle}{-2em}

  \title{California Housing Prices}
    \pretitle{\vspace{\droptitle}\centering\huge}
  \posttitle{\par}
    \author{Diego Alejandro Mernies}
    \preauthor{\centering\large\emph}
  \postauthor{\par}
      \predate{\centering\large\emph}
  \postdate{\par}
    \date{Primer cuatrimestre 2019}


\begin{document}
\maketitle

\section{Introducción}\label{introduccion}

Realizaremos una análisis de los datos de las casas que se encuentran en
un distrito determinado de \textbf{California} y algunas estadísticas
basadas en el censo de 1990.

\subsection{Objetivo}\label{objetivo}

Predecir el precio de las casa de la época con un modelo de regresión
lineal.

En primer lugar cargamos las librerías requeridas. Si no las tiene en su
sistema, puede instalarlas con \texttt{install.packages("librería")}.

\begin{Shaded}
\begin{Highlighting}[]
\KeywordTok{library}\NormalTok{(readr)}
\KeywordTok{library}\NormalTok{(dplyr)}
\KeywordTok{library}\NormalTok{(corrplot)}
\KeywordTok{library}\NormalTok{(ggplot2)}
\CommentTok{#library(rms)}
\end{Highlighting}
\end{Shaded}

\subsection{Definición de contantes}\label{definicion-de-contantes}

A continuación se define las \textbf{constantes} que se utilizarán en el
proyecto.

\begin{Shaded}
\begin{Highlighting}[]
\CommentTok{# URL donde reside el dataset a utilizar}
\NormalTok{dataurl <-}\StringTok{ "https://github.com/dmerniestic1987/tp_ciencia_datos_california_housing/blob/master/input/housing.csv"} 
\NormalTok{datadir <-}\StringTok{ "~/workspace/R/data"} \CommentTok{# Ubicación local en donde se descargará el dataset}
\end{Highlighting}
\end{Shaded}

\section{Carga de datos}\label{carga-de-datos}

\subsection{Set de datos}\label{set-de-datos}

El set de datos es un archivo .csv (comma separated value) de
exactamente 10 columnas y 20641 filas de las cuales la primera son los
nombres. El archivo de input original se llama \textbf{housing.csv} y se
tomó de {[}California Housing Price{]}
(\url{https://www.kaggle.com/camnugent/california-housing-prices}), pero
fue subido a un repositorio GIT para simplificar la descarga de los
datos y controlar las versiones.

Los datos pertenecen a las casas que se encuentran en un distrito de
California y algunas estadísticas basadas en los datos del censo de
1990. Las variables son:

\begin{longtable}[]{@{}ll@{}}
\toprule
\begin{minipage}[b]{0.13\columnwidth}\raggedright\strut
Variable\strut
\end{minipage} & \begin{minipage}[b]{0.17\columnwidth}\raggedright\strut
Descripción\strut
\end{minipage}\tabularnewline
\midrule
\endhead
\begin{minipage}[t]{0.13\columnwidth}\raggedright\strut
longitude\strut
\end{minipage} & \begin{minipage}[t]{0.17\columnwidth}\raggedright\strut
Qué tan lejos al oeste este está una casa. Un valor más alto está más al
oeste.\strut
\end{minipage}\tabularnewline
\begin{minipage}[t]{0.13\columnwidth}\raggedright\strut
latitude\strut
\end{minipage} & \begin{minipage}[t]{0.17\columnwidth}\raggedright\strut
Qué tan lejos al norte está una casa. Un valor más alto está más al
norte.\strut
\end{minipage}\tabularnewline
\begin{minipage}[t]{0.13\columnwidth}\raggedright\strut
housing\_median\_age\strut
\end{minipage} & \begin{minipage}[t]{0.17\columnwidth}\raggedright\strut
Edad media de una casa dentro de un bloque de casas. Un número más bajo
es un edificio más nuevo.\strut
\end{minipage}\tabularnewline
\begin{minipage}[t]{0.13\columnwidth}\raggedright\strut
total\_rooms\strut
\end{minipage} & \begin{minipage}[t]{0.17\columnwidth}\raggedright\strut
Número total de ambientes dentro de un bloque de casas.\strut
\end{minipage}\tabularnewline
\begin{minipage}[t]{0.13\columnwidth}\raggedright\strut
total\_bedrooms\strut
\end{minipage} & \begin{minipage}[t]{0.17\columnwidth}\raggedright\strut
Número total de habitaciones dentro de un bloque de casas.\strut
\end{minipage}\tabularnewline
\begin{minipage}[t]{0.13\columnwidth}\raggedright\strut
population\strut
\end{minipage} & \begin{minipage}[t]{0.17\columnwidth}\raggedright\strut
Número total de personas que residen dentro de un bloque de casas.\strut
\end{minipage}\tabularnewline
\begin{minipage}[t]{0.13\columnwidth}\raggedright\strut
households\strut
\end{minipage} & \begin{minipage}[t]{0.17\columnwidth}\raggedright\strut
Número total de hogares, un grupo de personas que residen dentro de una
unidad de hogar, por un bloque.\strut
\end{minipage}\tabularnewline
\begin{minipage}[t]{0.13\columnwidth}\raggedright\strut
median\_income\strut
\end{minipage} & \begin{minipage}[t]{0.17\columnwidth}\raggedright\strut
Ingreso promedio para hogares dentro de un bloque de casas (medido en
decenas de miles de dólares estadounidenses)\strut
\end{minipage}\tabularnewline
\begin{minipage}[t]{0.13\columnwidth}\raggedright\strut
median\_house\_value\strut
\end{minipage} & \begin{minipage}[t]{0.17\columnwidth}\raggedright\strut
Valor medio de la vivienda para hogares dentro de un bloque (medido en
dólares estadounidenses)\strut
\end{minipage}\tabularnewline
\begin{minipage}[t]{0.13\columnwidth}\raggedright\strut
ocean\_proximity\strut
\end{minipage} & \begin{minipage}[t]{0.17\columnwidth}\raggedright\strut
Ubicación de la casa con relación al oceano o mar.\strut
\end{minipage}\tabularnewline
\bottomrule
\end{longtable}

\subsection{Lectura de los datos}\label{lectura-de-los-datos}

\begin{Shaded}
\begin{Highlighting}[]
\KeywordTok{plot}\NormalTok{(cars)}
\end{Highlighting}
\end{Shaded}

\includegraphics{TP_Mernies_Diego_DataScience_CaliforniaHousing_files/figure-latex/unnamed-chunk-3-1.pdf}

Add a new chunk by clicking the \emph{Insert Chunk} button on the
toolbar or by pressing \emph{Ctrl+Alt+I}.

When you save the notebook, an HTML file containing the code and output
will be saved alongside it (click the \emph{Preview} button or press
\emph{Ctrl+Shift+K} to preview the HTML file).

The preview shows you a rendered HTML copy of the contents of the
editor. Consequently, unlike \emph{Knit}, \emph{Preview} does not run
any R code chunks. Instead, the output of the chunk when it was last run
in the editor is displayed.


\end{document}
